%% start of file `template_en.tex'.
%% Copyright 2007 Xavier Danaux (xdanaux@gmail.com).
%
% This work may be distributed and/or modified under the
% conditions of the LaTeX Project Public License version 1.3c,
% available at http://www.latex-project.org/lppl/.


\documentclass[11pt,a4paper]{moderncv}

% moderncv themes
%\moderncvtheme[blue]{casual}                 % optional argument are 'blue' (default), 'orange', 'red', 'green', 'grey' and 'roman' (for roman fonts, instead of sans serif fonts)
\moderncvtheme[blue]{classic}                % idem

% character encoding

\usepackage[utf8]{inputenc}                   % replace by the encoding you are using

% adjust the page margins
\usepackage[scale=0.852]{geometry}
\recomputelengths                             % required when changes are made to page layout lengths

% personal data
\firstname{Pedro}
\familyname{Arnedo}
\title{Senior Software Engineer}        % optional, remove the line if not wanted
\address{10 Conwell St.}{02143, Somerville, MA}    % optional, remove the line if not wanted
\mobile{(+1) 617.380.8837}                            % optional, remove the line if not wanted
% \phone{phone (optional)}                    % optional, remove the line if not wanted
% \fax{fax (optional)}                        % optional, remove the line if not wanted
\email{parnedo@gmail.com}                      % optional, remove the line if not wanted
% \extrainfo{additional information (optional)} % optional, remove the line if not wanted
\photo[40pt]{fotoCarnet.jpg}                % '64pt' is the height the picture must be resized to and 'picture' is the name of the picture file; optional, remove the line if not wanted
% \quote{Some quote (optional)}               % optional, remove the line if not wanted

\nopagenumbers{}                              % uncomment to suppress automatic page numbering for CVs longer than one page


%----------------------------------------------------------------------------------
%            content
%----------------------------------------------------------------------------------
\begin{document}
\maketitle

\hyphenation{Fa-cul-tad}
\section{Qualification Summary}
Experience encompassing various technologies including \textbf{C/C++(QT, stl, boost)}, \textbf{Python (Django, tornado) }, Java, REST, SOAP, ELK, AWS, vim, \textbf{backend development}, realtime large-scale web services, SQL, \textbf{Oracle}, Agile.

\begin{itemize}
    \item Over 8 years of hands-on experience in software development and IT
    \item Demonstrated abilities to build for scale and maintain top rated high volume products
    \item Startup oriented, quick learner, able to “own” the product, assume responsibilities, adapt to changes and/or pivot when needed 
    \item Able to coordinate efforts and guide teams in cross-cultural environments
\end{itemize}

\section{Trainings and certifications}
\cventry{2017}{Management Development for New managers/Supervisors}{Amadeus}{Boston}{}{}
\cventry{2015}{MongoDB certification}{Amadeus}{Nice}{}{}
\cventry{2014}{Scrum master certification}{Amadeus}{Nice}{}{}
\cventry{2013}{Oracle optimization}{Amadeus}{Nice}{}{}


\section{Professional Experience}
% \cventry{year--year}{Job title}{Employer}{City}{}{Description}  % arguments 3 to 6 are optional
\cventry{2014-Present}{Amadeus North America employee in Hotel department}{Senior Software Developer}{C++, Oracle, Python, Javascript}{}{
\begin{itemize}
	\item During the last three years, I've played a major role in the development of the rates management system for InterContinental. I've worked together with 20 other developers, 5 business analysts and 4 QAs in a scrum environment to deliver a solution that allows IHG to decommission a 50 years old system.
	\item \textbf{Recruitment:} Performed technical interviews with senior and junior candidates. Perform mentoring to Junior newcomers so they can become productive members of the team in a reduced amount of time \textit{recruitment, mentorship, pair programming}
	\item \textbf{Design:} As initial member of the team I've been in a privileged position to design the Rate Management System for IHG. I designed the system following an N-Tier architecture, separating the data access layer from the business layer.\textit{(N-tier, SOA)}
	\item \textbf{Testing:} I've set up policies to enforce the TDD for at least the data access layer part and enforce other policies like "no dynamic queries" generation. The framework used is google test and google mock. \textit{TDD, gmock, gtest} 
	\item \textbf{Development:} I've set up the initial C++ code architecture that allowed us to deliver new services ( currently we have 90) in a fast and coherent way. \textit{C++, protobuf}
	\item \textbf{Monitoring:} I've set up a system using ELK to monitor the performances of the rate management services. It tracks thanks to a homemade c++ library to track where the service is spending most of the time. \textit{Bash, ELK, Python} 
	\item \textbf{Troubleshooting:} Setup od a Django application that abstracts the database by showing all the data for a given object no matter what are the tables in which it's stored. \textit{Oracle, Django, JQuery, Bootstrap, d3js}
	\item \textbf{Management:} As a senior member of the team I've participated in exercises that required complex sizing and OOM in collaboration with the senior management of the division.  
	\item \textbf{Agile:} I've been scrum master for one year and a half in the team here in the US plus another year abroad in France.
\end{itemize}}
\cventry{2012-2016}{Creator of Scan\& android application}{New communication technology based on qr codes}{}{}{}%{www.scanand.com}{Currently on beta}{}
\cventry{2010-2014}{Amadeus S.A.S employee in Hotel department}{Software Developer}{}{}{
\begin{itemize}
	\item During the first four years of my career, I've been working on a myriad of projects, all of them under the scope of the hotel department.
	\item \textbf{Performance:} I've performed oracle performances studies for the reservation search services. I've performed a database study on indexes, data storage, and queries definition, leading to an increase of performance of several orders of magnitude.\textit{(Oracle, execution plan, clob, index-organized-table)}
	\item \textbf{Development:} I've consolidated all search and retrieve reservation services in AHP platform in a single service. Managing up to 23 versions of the same service in an organized, efficient and no-error-prone way. \textit{(TDD, Decorator pattern)}
	\item \textbf {Code refactoring:} In 2013 I extracted the search and retrieve reservation backend from a monolithic reservation component. An architecture that allows modularity, readability, extensibility was put in place as well as a design that facilitates the maintenance. 
    \item \textbf{Middleware:} Evoucher is a core middleware used by 5 different teams across the hotel department, a release notes system and documentation was put in place to improve communication.
	\item \textbf{Code audit:} in 2013 I performed an audit in a core middleware component of the hotel department. This audit was considered by the management as accomplished exceeding targets.
	\item \textbf{Working under pressure:} In 2012 I implemented a profile linking functionality on the profile management backend. Despite having to depend on other departments the project was delivered on time.
	\item \textbf{Bug fixing:} On 2011 I helped in the integration of Travelport, I focused on bug fixing helping to keep the project on track. As consequence, the bug backlog was drastically reduced.
\end{itemize}}
\cventry{2010}{Computer security internship at Beihang University}{Specialization in computer security}{Sensor Networks, distributed security}{}{}
\cventry{2009}{6-month internship "Refonte de l'IHM d'EGIPTE"}{Atos Origin}{Meylan}{}{Used technologies: C++, QT, boost, metaprogramming}
%\cventry{2007--2008}{6 months of internship at the "Software Ingeniering" department}{Facultad de Informática, U.P.M}{Madrid}{}{Used technologies: Php, css, xhtml, linux, LAMP}
% \cventry{2006--2008}{Wifi security, WEP weakness, how to secure your net,\dots}{Speaker}{Facultad de Informática, U.P.M}{Madrid}{ and courses at the UPM as member of ACM student chapter}
% \cventry{2005--2008}{Instant Messenger OpenSource library LibImMsnP}{Mantainer and main developer}{}{}{LibImMsnp belongs to kinapsis-libs, \htmladdnormallink{http://developer.berlios.de/projects/kinapsis-libs}{http://developer.berlios.de/projects/kinapsis-libs/}}
%
% \cventry{2006--2008}{Python programming, GUI and quick scripting}{Speaker}{Facultad de Informática, U.P.M}{Madrid}{talks and courses at the UPM as member of ACM student chapter}
% \cventry{2006}{Fourth edition of the CUPCAM programming contest}{Computer installation and configurations responsible}{Facultad de Informática, U.P.M}{Madrid}{\htmladdnormallink{http://www.fi.upm.es/cupcam2006}{http://www.fi.upm.es/cupcam2006/}}
% \cventry{2005--2007}{C language}{Speaker}{Facultad de Informática, U.P.M}{Madrid}{talks and courses at the UPM as member of ACM student chapter}
% \cventry{2004--2007}{Linux Operating System use and administration}{Speaker}{Facultad de Informática, U.P.M}{Madrid}{talks and courses at the UPM as member of ACM student chapter}

\section{Education}
\cventry{2009--2010}{Master II "Security, Cryptology and Coding of Information Systems"}{ENSIMAG}{Grenoble}{}{}
\cventry{2008--2010}{Double degree, speciality "Ingénierie des Systèmes d'Information"}{ENSIMAG}{Grenoble}{}{}
% \cventry{year--year}{Degree}{Institution}{City}{\textit{Grade}}{Description}  % arguments 3 to 6 are optional
\cventry{2001--2008}{Computer Engineering}{Universidad Politécnica de Madrid}{Madrid}{}{}
%\cventry{2007--2008}{Scholar at the UPM Software Engineering department}{Facultad de Informática, U.P.M}{Madrid}{}{Working with Php, css, xhtml, linux, LAMP}
%\cventry{2005--2006}{Scholar of the ACM student chapter association}{Facultad de Informática, U.P.M}{Madrid}{}{}
%\cventry{2004--2005}{Scholar of internet courses production and Tecnical Service at GATE}{Facultad de Informática, U.P.M}{Madrid}{}{}

\section{Languages}
% \cvlanguage{language 1}{Skill level}{Comment}
\cvlanguage{Spanish}{\small{Mother tongue}}{}
\cvlanguage{French}{\small{Bilingual}}{}
\cvlanguage{English}{\small{Fluent}}{}


\section{Some projects completed during my studies at UPM}
\cventry{2007-2008}{C compiler written in python language}{}{}{}{}
\cventry{2008}{}{Usability design and test of a class distributor for the Facultad de Informática - UPM}{}{}{}
%\cventry{2008}{Remote `dir` command for distributed course}{}{}{}{}
%\cventry{2007-2008}{Planification of a project following the TSPi standard}{}{}{}{One year practice with four other people}
%\cventry{2007}{Minikernel programming with basic functions}{}{}{}{}
%\cventry{2006}{}{Mathematical interpoling of GPS coordinates using real data (file .sp3)}{}{}{}
i%\cventry{2005-2006}{}{I/O polling and interruption programming in assembler into a MC68000 processor, RAM memory and a MC68681 DUART}{}{}{}
%\cventry{2003-2004}{}{Cache memory management in a 88110 processor}{}{}{}
%\cventry{2003}{}{Disperse matrix handling in 88110 Assembler}{}{}{}
\cventry{2003}{}{Sokoban game design and programming in ada95}{}{}{}
%\cventry{2002}{Microcode binary programming of intel i8080}{}{}{}{}
%\cventry{2002}{Minishell in C language (pipes, background, autocompletion, ...)}{}{}{}{}

%\section{Publications}
%\cventry{2007}{C programming language introduction guide}{Written with other ACM members}{}{}{\htmladdnormallink{http://acm.asoc.fi.upm.es/documentos/c2007/manual.pdf}{http://acm.asoc.fi.upm.es/documentos/c2007/manual.pdf}}
\section{Community Live}
\cventry{2007--2008}{President of the ACM student chapter association}{}{Facultad de Informática, U.P.M}{Madrid}{}
\cventry{2006--2007}{Vice-President of the ACM student chapter association}{}{Facultad de Informática, U.P.M}{Madrid}{}

\section{Other interesting information}
%\cventry{-}{Owner of driving license type B}{}{}{}{}
\cventry{-}{Amateur snowboard and amateur scuba diving}{Easy-going, responsible and reliable}{}{}{}
%\cventry{-}{Amateur photographer, work group and Linux lover}{}{}{}{}
%\cventry{-}{Easy-going, friendly, responsible and reliable}{}{}{}{}
%%\cventry{}{}{}{}{}{}
%% \section{Aficiones}
%% % \cvline{hobby 1}{\small Description}
%% \cvline{Guitarra}{\small Formación elemental tanto de Guitarra Clásica como de Solfeo}
%% \cvline{Pintura}{\small Asistencia a cursos de pintura durante un periodo de 5 años}
%% \cvline{Deportes}{\small Aficionado a todos los deportes. Practico paddle, fútbol y baloncesto}
\closesection{}                   % needed to renewcommands
\renewcommand{\listitemsymbol}{-} % change the symbol for lists

% \section{Extra 1}
% \cvlistitem{Item 1}
% \cvlistitem{Item 2}
% \cvlistitem[+]{Item 3}            % optional other symbol

% \section{Extra 2}
% \cvlistdoubleitem[\Neutral]{Item 1}{Item 4}
% \cvlistdoubleitem[\Neutral]{Item 2}{Item 5}
% \cvlistdoubleitem[\Neutral]{Item 3}{}

% Publications from a BibTeX file
% \nocite{*}
% \bibliographystyle{plain}
% \bibliography{publications}       % 'publications' is the name of a BibTeX file

\end{document}


%% end of file `template_en.tex'.
